\section{Test na hamiltonovskou $r$-cestu}
Testy uz�v�r� budeme prov�d�t pomoc� programu \emph{HamiltonPath}
s t�m, �e v definici uz�v�ru bude nastaven test na $r$-cestu
(\verb|test: sPath|).

\subsection{Uz�v�r $K_{4-4}-a$} \label{test:s:k4-4-a}
Prov��ujeme strukturu $K_{4-4}-A$. Podm�nky uz�v�ru navrhl �ada.
Zad�n� ze souboru \texttt{/Tests/\hyp path/\hyp s/\hyp
k4-4-a/\hyp closure.txt} je n�sleduj�c�.

\begin{ttfamily} \noindent
\\
nodes:  u,v,x,y1,y2 \\
path:   u,v \\
test:   sPath \\
\\
condition:       \{x,y1,y2\} <= N(u) \& N(v) \\
\\
condition:        N(x) <= N[u] | N[v] \\
condition: N(y1) \bs N[x] <= clique \\
condition: N(y2) \bs N[x] <= clique \\
\end{ttfamily}

\noindent Struktura je vyobrazena na obr�zku \ref{fig:k-a-b} pod
p�smenem A. V�stup programu je um�st�n v souboru
\texttt{/Tests/\hyp path/\hyp s/\hyp k4-4-a/\hyp nohup.out}.

\subsubsection{V�sledky testu}
Uz�v�r testem pro�el za 1626 sekund na procesoru P450.
Vygeneroval 756 v�jimek z 718128 testovan�ch graf�. Proto z
pohledu tohoto programu uz�v�r nezachov�v� vlastnost neexistence
hamiltonovsk� $r$-cesty.
