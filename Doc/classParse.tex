\subsection{Dokumentace t��dy Parse}
\label{classParse}\index{Parse@{Parse}}
{\tt \#include $<$Parse.h$>$}

T��da zjednodu�uje pr�ci s �et�zci obsahuj�c�mi v�razy. Pomoc�
metody {\bf Remove\-WS} odstra�uje nadbyte�n� b�l� znaky z
textov�ho �et�zce. Metoda {\bf Split} umo��uje snadn� rozd�len�
�et�zce v p��pad�, �e obsahuje dan� oper�tor. Metoda zaji��uje
spr�vn� zpracov�n� z�vorek. Metoda {\bf
Extract\-Parameter\-From\-Function} z�sk� parametry funkce
p�edan� jako parametr metody.

\obrazekparam{width=223pt}{obrazky/prog/classParse__inherit__graph}
{fig:classParse__inherit__graph}{Diagram d�di�nosti pro t��du
Parse}

\subsubsection*{Ve�ejn� metody}
\begin{CompactItemize}
\item
bool {\bf Remove\-WS} (string \&str)
\item
bool {\bf Split} (string str, string s\-Delimeter, string \&s\-Begin, string \&s\-Rest, bool b\-Test=false)
\item
bool {\bf Split} (string str, set$<$ string $>$ ss\-Delimeters, string \&s\-Begin, string \&s\-Rest, string \&s\-Used\-Delimeter)
\item
bool {\bf Remove\-Ambient\-Brackets} (string \&str, char ch\-Bracket1, char ch\-Bracket2)
\item
bool {\bf Extract\-Parameter\-From\-Function} (string str, string s\-Function, char ch\-Bracket1, char ch\-Bracket2, string \&s\-Parameter)
\item
bool {\bf Successor} (string str, string \&s\-Node)
\item
bool {\bf Predecessor} (string str, string \&s\-Node)
\end{CompactItemize}
